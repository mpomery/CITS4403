\section{Exercises}

\par
Below are a list of exercises that will help you to explore \texttt{rentals.py},  understand how it works and extend it.

\begin{excercise}
Obtain a copy of \texttt{rentals.py} and modify it so that it runs for 100 years. Be patient as it will take a while to run. Change it back to 10 years when you are done.
\end{excercise}

\begin{excercise}
Run the program with a different city size using the command line arguments and compare the outputs.
\end{excercise}

\begin{excercise}
Determine the Big-O notation for each step.
\end{excercise}

\begin{excercise}
Give people an income at random, weighted so that some people are low income earners, most people are medium income earners and a few people are high income earners.
\end{excercise}

\begin{excercise}
Determine if using `@property` causes significant performance degradation by removing it from classes where possible.
\end{excercise}

\begin{excercise}
Modify the city so that there is a hole in the center. Run the simulation and see how this changes the output.
\end{excercise}

\begin{excercise}
Implement the calculate happiness function so that it is based on how much rent is compared to income, where it is compared to their current house and how many neighbors it has.
\end{excercise}