\section{Exercises}

\par
Below are a list of exercises that will help you to explore \texttt{income.py},  understand how it works and extend it.

\begin{excercise}
Obtain a copy of \texttt{income.py} and run it. Modify it so that it runs for 100 steps and compare the two outputs. Be patient as it will take a while to run.
\end{excercise}

\begin{excercise}
Run the program with a different city size using the command line arguments and compare the outputs.
\end{excercise}

\begin{excercise}
Determine the Big-O notation for each step.
\end{excercise}

\begin{excercise}
Give people an income at random, weighted so that some people are low income earners, most people are medium income earners and a few people are high income earners.
\end{excercise}

\begin{excercise}
Determine if using `@property` causes significant performance degradation by removing it from classes where possible.
\end{excercise}

\begin{excercise}
Modify the city so that there is a hole in the center. Run the simulation and see how this changes the output.
\end{excercise}
