\section{Introduction}

% What are agent based models? 
\par
Finding a place to live in while renting can be very difficult. Expensive places all seem to be grouped together, right next to each other and all occupied while cheaper places are on the outskirts of town and relatively empty.  A good way to look at why this happens is by using agent based models.

% OR was it the very first
\par
Simulating population movements using an agent based model is nothing new. Thomas C. Schelling's "Dynamic Models of Segregation", published in 1971, was one of the first to do this. His model had two agents that both behaving the same way. They would move if they were "unhappy" because they were outnumbered by the other agent.

\par
Simulating a city of people moving around to houses with changing rent prices is non-trivial. You need to manage several variables such as rent and occupancy, as well as determine who is going to move into each house at the end of each month. 

\par
- Looking at rental prices \\
- By simulating people \\
- Seeing why expensive houses are close together \\

\par
- Simulating population movements using agent based models is nothing new. \\
- Thomas C. Schelling published "Dynamic Models of Segregation" in 1971 \\
  - Two Agents, both acted the same way \\
  - Moved around grid system preferring to not be outnumbered by other agent \\
- This model has two agents, People and the Houses they live in. \\
- Each agent has its own rules and interacts with each other \\

