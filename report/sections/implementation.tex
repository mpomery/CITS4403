\section{Implementation}

\par
To simulate a city with people moving around it we needed to create the environment and the agents. The simulation has been implemented in \texttt{income.py} and is based off of \texttt{rentals.py}, an incomplete agent based model. The model is of people moving around if their incomes are too different from the people in their neighborhood.

\subsection{Named Tuples}

In all programming, making the code easy to read and understand is important. One way to do this is through the use of Python's named tuples. They allow the creation of unchangeable tuples with properties that can be accessed by name instead of index. They easy to use and make code significantly more readable. The main named tuple used in the simulation is Coordinates, which are passed between several objects. Named tuples make it possible to refer to the x and y coordinates as \texttt{Coordinates.x} and \texttt{Coordinates.y} instead of \texttt{Coordinates[0]} and \texttt{Coordinates[1]}.

\begin{code}[language=Python]
import collections

# Define a named tuple
Coordinates = collections.namedtuple('Coordinates', 'x y')
# Create a tuple
location = Coordinates(4, 5)
print(location.x) # Will print 4
print(location.y) # Will print 5
\end{code}

\subsection{City}

\par
The \texttt{City} object keeps track of where every person is and who live in their neighborhood. Determining who lives in a neighborhood is achieved by iterating through all cells around the node and returning the coordinates of any cell that contains another person.

\begin{code}[language=Python]
def get_neighbourhood(self, coords):
	neighbours = []
	for x in range(coords.x - self.neighbourhood_size, coords.x + self.neighbourhood_size + 1):
	    for y in range(coords.y - self.neighbourhood_size, coords.y + self.neighbourhood_size + 1):
	        if coords.x != x and coords.y != y:
	            neighbour_coords = Coordinates(x, y)
	            neighbour = self.get_house(neighbour_coords)
	            if neighbour != None:
	                neighbours.append(neighbour)
	return neighbours
\end{code}

\subsection{People and Population}

\par
\texttt{People} in the simulation share properties with real life people, but have everything unimportant to the simulation stripped away. They have an income, know what city they live in and know what their address is. Everything else such as name, age, gender, height, life goals, ambitions and sentience is unimportant to the simulation and so is left out. In agent based modeling you do not need to replicate every property of the being you are investigating, and can choose what you need.

Managing all of the \texttt{People} is the responsibility of the \texttt{Population} class. Each step \texttt{Population} gets a list of all the empty houses and all the people who want to move and moves as many people as it can to a new random position.

\subsection{CityPrinter}

\texttt{CityPrinter} is used to create a colour heat map of the incomes of the population and where they live. It uses the \texttt{wx} Python Package to create the graphics. To create this image, a colour is generated for every cell on the map that a person lives on, which is then added to a canvas. Once every point has been added to canvas, the frame is shown and the application runs until the frame is closed.

\begin{code}[language=Python]
def create_heatmap(self):
	min_income = self.population.min_income
	max_income = self.population.max_income
	print("Lowest Income: " + str(min_income))
	print("Highest Income: " + str(max_income))
	
	for house in self.city.houses:
	    x = house.x - self.city.size / 2
	    y = house.y - self.city.size / 2
	    person = self.city.get_house(house)
	    if person != None:
	        col = self.color(person.income, min_income, max_income)
	        self.canvas.AddPoint((x, y), Color = col)
	
	self.frame.Show()
	self.app.MainLoop()
\end{code}

\par
Determining the colour for a specific income is done using the \texttt{color} method. This method approximates the sequential colour scheme from the Department of Geography at the University of Oregon\cite{dog-uoo}. It does this by transferring the value from the scale provided to the red, green and blue scales in the method. It returns an object that can be used right away in the heat map creation methods.

\begin{code}[language=Python]
def color(self, value, min_val, max_val):
    # Approximating http://geog.uoregon.edu/datagraphics/color/Bu_10.txt on the fly
    red_range = (0, 0.9)
    green_range = (0.25, 1.0)
    blue_range = (1.0, 1.0)
    
    percentage_of_range = 1 - (value - min_val)/(max_val - min_val)
    
    red = (((red_range[1] - red_range[0]) * percentage_of_range) + red_range[0]) * 255
    green = (((green_range[1] - green_range[0]) * percentage_of_range) + green_range[0]) * 255
    blue = (((blue_range[1] - blue_range[0]) * percentage_of_range) + blue_range[0]) * 255
    
    return wx.Colour(red, green, blue, 1)
\end{code}




