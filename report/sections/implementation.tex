\section{Implementation}

\par
To simulate a city with people moving around it we need to create two agents, the houses themselves and the people who live in them. We then create a city for these houses to be built in and a population that live in this city. These two groups then interact with each other and act depending on the state of the other. At the end of the simulation there are two things we can look at, rental prices and income distribution.

Since the system we are creating is complex, making the code easy to read and understand is important. One way to do this is through the use of pythons named tuples. They allow the creation of unchangeable tuples with properties that can be accessed by name instead of index. They easy to use and make code significantly more readable.

\begin{lstlisting}[language=Python]
import collections

# Define a named tuple
Coordinates = collections.namedtuple('Coordinates', 'x y')

# Create a tuple
location = Coordinates(4, 5)

# Will print 4
print(location.x)

# Will print 5
print(location.y)
\end{lstlisting}

\subsection{Agents}

\subsubsection{House}
\par
- Have a price and possibly an occupant \\
- Increase rent after certain amount of time \\
- Decrease rent to try and get people to rent it \\

\subsubsection{People}
\par
- Similar to real people \\
  - Have an income \\
  - Have a bank balance \\
  - Pay rent every month \\
  - Can only move when rent agreement finishes \\
- Unlike real world \\
  - No share houses \\
  - No income increases \\
  - Don't have sentience \\

\subsection{Managing the Agents}
\subsubsection{City}
\par
- Manages all the houses \\
- Makes sure rent changes \\

\subsubsection{Population}
\par
- Manages all the people \\
- Makes sure they are happy \\
- Make sure they move if they aren't \\
- Each step updates rent of old houses, moves people, then updates rent of occupied houses \\