\section{Analysis}

\par
Choosing to attempt to implement the complex social system that is rental properties and a society that lives in them was possibly not the best idea. In the real world there are several factors that affect the price of rental properties such as age of the house, and distance to public transport, shopping centers, workplaces, schools and major population centers. People choose houses based on more than how far away it is from their current place and how much the rent is, as they look for community groups, where their friends and family live, and how many bedrooms, bathrooms and toilets it has.

\par
Some fatal flaws exist in the simulation, such as the fact that rental properties can be driven all the way down to \$0. However, the simulation does display some interesting things however. High cost rental properties start to cluster and get filled by high income earners. The further away you move from these clusters, the lower the rent and incomes become.

\par
One way to simplify and improve the simulation is to remove the aspect of rental prices altogether and have people moving based one income. A mock up of this idea can be seen in \texttt{income.py}.